\subsection{Search execution and reporting}\label{subsec:search-execution-and-reporting}

After finishing the preparations, the next step is to identify papers that should provide a basis for the review.
The first step is to \textbf{perform the searches} in search engines.
Afterwards, the\ \textbf{search results} (search date, search areas, number of results) \textbf{should be documented}.

Simultaneously with the search, literature is going to be \textbf{tentatively screened for inclusion}, based on titles and abstracts of works.
This stage does away with formal selection criteria -- the papers are chosen for further consideration based on a brief intuitive judgement of suitability and relevance;
anything irrelevant to the discussion is meant to be discarded during the second screening for inclusion.
In this stage, the decision is made after reading the papers' \textbf{introduction and conclusions} sections.
This time, the criteria for exclusion should be \textbf{made explicit and documented}.

After filtering out the least interesting texts, the last screening step aims to identify other articles
that should not be included in the review \textbf{based on their full content}.
Reasons for rejecting papers in this stage also should be provided.

The last step of the search is performing a backward and forward search through the references and citations of papers chosen for the review.
It aims to complement the results of database searches with other relevant papers that would not otherwise be found.

\todo[inline]{przebieg przegladu}
