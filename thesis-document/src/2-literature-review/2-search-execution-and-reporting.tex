\subsection{Search execution and reporting}\label{subsec:search-execution-and-reporting}

After finishing the preparations, the next step is to identify papers that should provide a basis for the review.
The first step is to \textbf{perform the searches} in search engines.
Afterwards, the\ \textbf{search results} (search date, search areas, number of results) \textbf{should be documented}.

Simultaneously with the search, literature is going to be \textbf{tentatively screened for inclusion}, based on titles and abstracts of works.
This stage does away with formal selection criteria -- the papers are chosen for further consideration based on a brief intuitive judgement of suitability and relevance;
anything irrelevant to the discussion is meant to be discarded during the second screening for inclusion.
In this stage, the decision is made after reading the papers' \textbf{introduction and conclusions} sections.
This time, the criteria for exclusion should be \textbf{made explicit and documented}.

After filtering out the least interesting texts, the last screening step aims to identify other articles
that should not be included in the review \textbf{based on their full content}.
Reasons for rejecting papers in this stage also should be provided.

The last step of the search is performing a backward and forward search through the references and citations of papers chosen for the review.
It aims to complement the results of database searches with other relevant papers that would not otherwise be found.

\todo[inline]{przebieg przegladu}

inclusion criteria:
\begin{itemize}
    \item discussion of expressiveness of generating UI
    \item review of UIDLs
    \item MBUID approaches
    \item description/specification of UI representation
    \item UIDLs
\end{itemize}

\begin{itemize}
    \item two searches in scopus and web of science
    \item 6-9th of February 2023
    \item scopus 1414 documents
    \item web of science - 856
\end{itemize}

first stage: loose and tentative screening for inclusion, based on titles and abstracts:
\begin{itemize}
    \item scopus: 133 documents
    \item web of science: 77 documents
    \item after removing duplicates: 154 documents
\end{itemize}

second stage: exclusion of least relevant papers, still based on titles and abstracts
\begin{itemize}
    \item 107 papers retained
    \item 47 discarded
    \item \todo{formulate more precisely}discarding criteria:
    \begin{itemize}
        \item in general, papers not relevant to the task at hand
    \end{itemize}
\end{itemize}

third stage: further elimination of irrelevant papers, loosely based on full text
\begin{itemize}
    \item 45 papers retained
    \item 62 papers discarded
    \item \todo{formulate more precisely}discarding criteria
    \begin{itemize}
        \item development a MBUIDE
        \item focus outside the task at hand: interaction modelling, UI or general development, not model-based
        \item relying on other paper, or an existing description
        \item lack of clear UI description or a description for a specific use case
        \item solution for a more specific problem in MBUID (mostly adaptation)
    \end{itemize}
\end{itemize}

fourth stage: further elimination of irrelevant papers, more thorough reading of full text
\begin{itemize}
    \item 11 papers retained
    \item 34 papers discarded
    \item \todo{formulate more precisely}discarding criteria:
    \begin{itemize}
        \item lack of a clear UI description
        \item low quality, simple model, informal description
        \item lack of a concrete UI description (only AUI)
        \item referring to a different paper or an existing description
    \end{itemize}
\end{itemize}

snowballing: reading papers that were referred to in stage 4 and backward and formward references in chosen papers
resulted in four extra papers
