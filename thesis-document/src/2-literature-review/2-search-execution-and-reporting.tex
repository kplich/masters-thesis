\subsection{Search execution and reporting}\label{subsec:search-execution-and-reporting}

After finishing the preparations, the next step is to \textbf{perform the searches} in the engines and \textbf{document the results} (search date, search areas, number of results).
The searches were performed from February 6$^{th}$ to February 9$^{th}$, 2023.
Scopus search returned 1414 results;
Web of Science search returned 856 documents.

While performing the search, the literature was \textbf{tentatively screened for inclusion} based on titles and abstracts of works.
Papers were chosen if they:
\begin{itemize}
    \item seemed to present a representation of user interface/a UIDL,
    \item seemed to present an approach to MBUID that might use a UI representation,
    \item seemed to present a review of UI representations,
    \item seemed to discuss the problem of expressiveness of UI representations.
\end{itemize}
The screening was rather broad and not too precise, so that as much relevant work as possible is identified;
papers unrelated to the problem stated would be discarded during later stages of the review.
This stage produced 133 documents found in Scopus and 77 documents found in Web of Science.
After merging the two sets and removing duplicate papers, 154 documents were left for further consideration.

In the second stage of the review, the set of papers was reviewed again in order to eliminate documents not connected to the subject of the thesis.
The judgement was additionally based on the papers' introductions and conclusions.
The process resulted in discarding of 47 papers, mostly due to focus on problems outside the area of MBUID\@.
Other reasons for exclusion included:
\begin{itemize}
    \item papers concerning functional/declarative development of UIs,
    \item papers describing methods for generation of full applications, not focused on UIs
    \item papers based on, evaluating or comparing existing UI representations.
    \item papers without a clear definition of a UI representation
\end{itemize}

The third stage of the review eliminated further papers, based on skimming the full text of the papers.
In this stage, 62 papers were discarded:
\begin{itemize}
    \item papers unrelated to MBUID, focused on UI or general development,
    \item papers related to problems adjacent to the thesis problem: interaction modelling, development of a MBUID environment, techniques for UI adaptation,
    \item papers relying on other paper or an existing UI description,
    \item papers without a clear definition of a UI description (or with a definition suited for a specific use case),
    \item unavailable papers.
\end{itemize}

After filtering out the least interesting texts, the last screening step aims to identify other articles
that should not be included in the review \textbf{based on their full content}.
Reasons for rejecting papers in this stage also should be provided.

The last step of the search is performing a backward and forward search through the references and citations of papers chosen for the review.
It aims to complement the results of database searches with other relevant papers that would not otherwise be found.

\todo[inline]{przebieg przegladu}

inclusion criteria:
\begin{itemize}
    \item discussion of expressiveness of generating UI
    \item review of UIDLs
    \item MBUID approaches
    \item description/specification of UI representation
    \item UIDLs
\end{itemize}

\begin{itemize}
    \item two searches in scopus and web of science
    \item 6-9th of February 2023
    \item scopus 1414 documents
    \item web of science - 856
\end{itemize}

first stage: loose and tentative screening for inclusion, based on titles and abstracts:
\begin{itemize}
    \item scopus: 133 documents
    \item web of science: 77 documents
    \item after removing duplicates: 154 documents
\end{itemize}

second stage: exclusion of least relevant papers, still based on titles and abstracts
\begin{itemize}
    \item 107 papers retained
    \item 47 discarded
    \item \todo{formulate more precisely}discarding criteria:
    \begin{itemize}
        \item in general, papers not relevant to the task at hand
    \end{itemize}
\end{itemize}

third stage: further elimination of irrelevant papers, loosely based on full text
\begin{itemize}
    \item 45 papers retained
    \item 62 papers discarded
    \item \todo{formulate more precisely}discarding criteria

\end{itemize}

fourth stage: further elimination of irrelevant papers, more thorough reading of full text
\begin{itemize}
    \item 11 papers retained
    \item 34 papers discarded
    \item \todo{formulate more precisely}discarding criteria:
    \begin{itemize}
        \item lack of a clear UI description
        \item low quality, simple model, informal description
        \item lack of a concrete UI description (only AUI)
        \item referring to a different paper or an existing description
    \end{itemize}
\end{itemize}

snowballing: reading papers that were referred to in stage 4 and backward and formward references in chosen papers
resulted in four extra papers
