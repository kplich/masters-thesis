\section{Literature review}\label{sec:literature-review}

methodology based on:

\todo[inline]{add references}

Guidance on Conducting a Systematic Literature Review~\cite{xiao_guidance_2019}

Systematic Literature Review in Computer Science \textendash A Practical Guide~\cite{neiva_systematic_2016}

Guidelines for Performing Systematic Literature Reviews in Software Engineering~\cite{kitchenham_guidelines_2007}

will cover three phases, covered in subsequent sections:
\begin{itemize}
    \item preparation and planning
    \item execution
    \item conclusion
\end{itemize}

\subsection[Planning and preparation]{Planning and preparation for review}\label{subsec:planning-and-preparation-for-review}

\begin{itemize}
    \item preparation for actual searching in databases
    \item justification and search for existing reviews
    \item choice of search engines
    \item final selection of keywords and general search strings
\end{itemize}

\subsubsection{Research questions}

\paragraph{RQ 1: What approaches are used for modelling graphical user interfaces?}
The premise of the work is to transform graphical data into an abstract representation;
its choice is an extremely important one as it might impact the whole solution

\paragraph{RQ 2: How can a graphical representation of GUI be converted to a machine-readable one?}
How do others do it, what initial and final representations do they use, what are the results, strengths, and limitations

\subsubsection{Keywords}
keywords extracted from the questions, with synonyms or related terms

\paragraph{RQ 1}

\begin{itemize}
    \item graphical user interface
    \begin{itemize}
        \item GUI
        \item UI
        \item user interface
        \item graphical interface
    \end{itemize}
    \item modelling
    \begin{itemize}
        \item (abstract/platform-independent) (meta-)model
        \item (abstract/platform-independent) representation
        \item modelling language
        \item specification
    \end{itemize}
\end{itemize}

\paragraph{RQ 2}

\begin{itemize}
    \item graphical user interface \todo{missing keywords from thesis title}
    \begin{itemize}
        \item GUI
        \item UI
        \item user interface
        \item graphical interface
    \end{itemize}
    \item convert
    \begin{itemize}
        \item transform(ation)
        \item rewrite/rewriting
        \item recreate/recreating
        \item generate/generation
    \end{itemize}
    \item machine(-readable)
    \begin{itemize}
        \item code
        \item implementation
    \end{itemize}
\end{itemize}

\subsubsection[Justification]{Justification for the review}
\begin{itemize}
    \item identify the state of the art
    \item point out gaps and further the body of knowledge
    \item build upon relevant and useful knowledge
    \item inform the architecture/techniques/solutions used in the method
    \item discussion of any existing reviews (and their effect on the need for another review)
\end{itemize}

\paragraph{RQ 1}
based on the keywords above, no systematic review of the topic has been found (on 4.02.2022)
J. Vanderdonckt \todo{missing reference} performed several semi-formal reviews of
user interface description languages over the years;
the authors describe the criteria of evaluation/description of languages but document it rather scantly;
they claim to attempt to describe a UI for a simple application, however no code or UI examples are provided;
moreover, the last review seems to have been published in 2010, that is more than 10 years from now;
this justifies another review

\begin{itemize}
    \item A Review of XML-compliant User Interface Description Languages (2003)
    \item A Theoretical Survey of User Interface Description Languages - Preliminary Results (2009)
    \item A Theoretical Survey of User Interface Description Languages - Complementary Results (2010)
\end{itemize}

the articles provide references that might serve as a benchmark for the effectiveness of the review \todo{intention not clear}
based on the review, 13 papers about UIDLs \todo{abbreviation not explained}
described are going to be included in the review;
papers not found through the primary search are going to be added *before* the first screening of inclusion;
additional keyword/term identified:
\begin{itemize}
    \item \enquote{user interface description language}/\enquote{UIDL}
    \item \enquote{multi-platform}/\enquote{multi-device}
    \item \enquote{platform-independent}/\enquote{device-independent}
\end{itemize}

\paragraph{RQ 2}
--- \todo{look for an SLR}

\subsubsection{Choice of search engines}
for the literature review, the following search engines will be used:
\begin{itemize}
    \item Scopus\footnote{\url{https://scopus.com}}
    \item IEEE Xplore\footnote{\url{https://ieeexplore.ieee.org}}
    \item ACM Digital Library\footnote{\url{https://dl.acm.org/}}
    \item Springer Link\footnote{\url{https://link.springer.com/}}
    \item DBLP\footnote{\url{https://dblp.uni-trier.de/}}
    \item Google Scholar\footnote{\url{https://scholar.google.com}}
    \item Semantic Scholar\footnote{\url{https://semanticscholar.org}}
\end{itemize}

\subsubsection{General search strings}

\paragraph{RQ 1}

separation of actually three specific parts (and inclusion of the UIDL term):
\begin{itemize}
    \item GUI
    \item abstraction from a platform/device
    \item description using a model
\end{itemize}

\begin{listing}
    \begin{minted}{text}
(
    ("graphical user interface" OR "user interface"
        OR "graphical interface" OR "UI" OR "GUI")
    AND
    ("abstract" OR "platform-independent" OR "device-independent"
        OR "multi-platform" OR "multi-device")
    AND
    ("language" OR "modelling language"
        OR "specification" OR "model" OR "meta-model"
        OR "representation" OR "description")
)
OR
(
    "UIDL" OR "user interface description language"
)
    \end{minted}
\end{listing}

the string will be adjusted to each of the search engines' syntax

\paragraph{RQ 2}
\todo[inline]{specify general search strings}

\subsubsection{Additional search criteria}

additional search criteria, to be used where possible:
\begin{itemize}
    \item papers published in and after 2000 \textendash with the emergence of the widely available Web and mobile devices \todo{is the timing correct here?}, the need for portable user interfaces should become apparent in the number of papers;
    while a 20-year span might include now-obsolete solutions, it might be necessary (as shown by the existing UIDL report) to gain a complete picture of the domain
    \item papers only related to computer science/software engineering/etc.
    \item papers primarily in English;
    German or Polish acceptable if need be
\end{itemize}

\subsubsection[Protocol]{Protocol (further steps, executed separately for each RQ)}
\begin{itemize}
    \item perform searches (adjust the search strings for each database)
    \item document search results (search base, search date, number of items, exact search string)
    \item refine search strings, if the need arises
    \item first screening for inclusion; based on titles and abstracts, introductions and conclusions
    \item possible exclusion criteria:
    \item paper not relevant enough
    \item paper too vague
    \item document any other criteria/reasons for exclusion
    \item second screening for inclusion based on full text
    \item data extraction and analysis (specify more thoroughly later?)
\end{itemize}

\subsection{Review process - RQ1}\label{subsec:review-process---rq1}
\todo[inline]{to perform}

\subsection{Conclusions}\label{subsec:conclusions}
\todo[inline]{to write}
