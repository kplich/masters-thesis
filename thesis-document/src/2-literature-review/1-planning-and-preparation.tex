\subsection[Planning and preparation]{Planning and preparation for review}\label{subsec:planning-and-preparation-for-review}

This section describes measures undertaken in preparation for the search of relevant literature and its review \textendash\ including formulation of research questions and search strings.

\subsubsection{Research questions}

\todo[inline]{elaborate the RQs more}

\paragraph{RQ 1: What approaches are used for modelling graphical user interfaces?}
The premise of the work is to transform graphical data into an abstract representation;
its choice is therefore an extremely important one, as it might impact the whole solution.

\paragraph{RQ 2: How can a graphical representation of GUI be converted to a machine-readable one?}
What methods are used?
What are their results, strengths, and limitations?
What initial and final representations do these methods use?

\subsubsection[Justification]{Justification for the review}

This review aims to \textbf{describe the state of the art} in the field, point out its gaps, and \textbf{demonstrate possible solutions} that can be used to address the thesis problem.
To provide background for the literature review, existing reviews
relevant to previously formulated research questions are discussed below.

\paragraph{RQ 1}
Based on the keywords above, no systematic review of the topic has been found (as of 4.02.2022.)
Vanderdonckt et al.\ performed several~\cite{souchon_review_2003,guerrero_garcia_theoretical_2009,guerrero_garcia_theoretical_2011}
semiformal reviews of \emph{user interface description languages} (UIDLs) over the years.
The authors describe the criteria of evaluation and description of UIDLs but document them rather scantly.
Although they claim to attempt to describe a UI for a simple application, no code or UI examples are provided.
Moreover, the last review seems to have been published in 2010 \textendash\ more than 10 years from now;
this justifies another review that would take into account more recent developments within the domain.
13 papers referenced in the three reviews mentioned are going to be included in the following review
\emph{before} the second screening for inclusion (based on introductions and conclusions);
they might serve as a benchmark for its effectiveness (they should also show up in the results of the primary search).

\paragraph{RQ 2}
\todo[inline]{look for an SLR}

\subsubsection{Keywords}
The keywords have been extracted from the proposed research questions
and have been supplemented with their synonyms or related terms.

\paragraph{RQ 1}

\begin{itemize}
    \item \textbf{graphical user interface}/user interface/graphical interface/GUI/UI
    \item \textbf{modelling}/(meta-)model/representation/(modelling) language/specification
    \item \textbf{multi-platform}/multi-device/platform-independent/device-independent
    \item \textbf{user interface description language}/UIDL
\end{itemize}

\paragraph{RQ 2}

\begin{itemize}
    \item \textbf{graphical user interface}/GUI/UI/user interface/graphical interface
    \item \textbf{sketch}/prototype/mockup
    \item \textbf{convert(-sion)}/transform(ation)/rewrite(-ing)/recreate(-ing/-ion)/generate(-ing/-ion)
    \item \textbf{representation}/code
\end{itemize}

\subsubsection{Choice of search engines}

The following search engines will be used:
\begin{itemize}
    \item Scopus\footnote{\url{https://scopus.com}}
    \item IEEE Xplore\footnote{\url{https://ieeexplore.ieee.org}}
    \item ACM Digital Library\footnote{\url{https://dl.acm.org/}}
    \item Springer Link\footnote{\url{https://link.springer.com/}}
    \item DBLP\footnote{\url{https://dblp.uni-trier.de/}}
    \item Semantic Scholar\footnote{\url{https://semanticscholar.org}}
\end{itemize}
The engines listed are well-accepted in the scientific community and index a wide variety of publications.

\subsubsection{General search strings}

Based on keywords from the previous sections, general search strings are defined.
They will not be used directly in queries but will provide a basis for engine-specific queries.

\paragraph{RQ 1}

The search string (shown in Listing~\ref{lst:general-search-string-rq-1})
comprises two complementary parts connected by an \texttt{OR} operator.
The first one is a conjunction of three more specific parts:
\begin{itemize}
    \item a set of terms relating to graphical user interfaces
    \item a set of terms describing abstracting from a platform/device
    \item a set of terms about modelling/describing
\end{itemize}
The second part mentions UIDLs directly (the term does not fit neatly into the other part of the string.)

\begin{listing}
    \caption{The general search string for RQ 1}
    \begin{minted}{text}
(
    ("graphical user interface" OR "user interface"
        OR "graphical interface" OR "UI" OR "GUI")
    AND
    ("abstract" OR "platform-independent" OR "device-independent"
        OR "multi-platform" OR "multi-device")
    AND
    ("language" OR "modelling language"
        OR "specification" OR "model" OR "meta-model"
        OR "representation" OR "description")
)
OR
(
    "UIDL" OR "user interface description language"
)
    \end{minted}
    \label{lst:general-search-string-rq-1}
\end{listing}

\paragraph{RQ 2}
\todo[inline]{specify general search strings}

\subsubsection{Additional search criteria}

To improve and narrow down the results of the searches, additional criteria are to be used:
\begin{itemize}
    \item papers ought to be related only to computer science/software engineering/etc.
    \item papers should be published not earlier than in 2000
    \item papers should be written in English
\end{itemize}

The restriction of the publication date is\todo{is the timing correct here?} motivated by the emergence of the widely available Web and mobile devices;
the need for portable user interfaces should become apparent in the number of relevant papers.
While a 20-year span might include now-obsolete solutions, it might be necessary
(as exemplified by the previously mentioned UIDL report) to gain a complete picture of the domain.
