\subsection[Planning and preparation]{Planning and preparation for review}\label{subsec:planning-and-preparation-for-review}

\subsubsection{Defining research questions and justifying the review}
To gather data for evaluation, three research questions need to be answered:
\begin{itemize}
    \item RQ1: What abstract representation of user interfaces exist?
    \item RQ2: To what extent these representations adhere to the Cameleon Reference Framework? \textit{(evaluation of comprehensiveness)}
    \item RQ3: To what extent these representations allow for a detailed specification of a UI? \textit{(evaluation of fidelity)}
\end{itemize}

There have been several surveys of UIDLs conducted in the past~\cite{souchon_review_2003, guerrero_garcia_theoretical_2009, guerrero_garcia_theoretical_2011, Jovanovic2013}.
However, they’re not systematic and well-documented.
The latest one has been published in 2013 and there does not seem to exist a more recent overview.
Additionally, the documents have been focused on XML-based languages and did not research any other forms of representing user interfaces.

There exists a systematic review of MBUID environments~\cite{Ruiz2018}.
While it provides a structured evaluation in the context of model-driven development and lists comprehensiveness and design flexibility as evaluation criteria, it does not focus on languages/representations themselves.
Additionally, it also does not include approaches from later than 2015.

\subsubsection{Defining search criteria}
First, relevant keywords and synonyms were first identified:
\begin{itemize}
    \item \textbf{abstract}/universal/generic/platform-independent/etc. \textit{(terms related to non-direct descriptions)}
    \item \textbf{user interface}/UI \textit{(terms related to user interfaces)}
    \item \textbf{representation}/(meta-)model/language/specification/etc. \textit{(terms related to descriptions)}
\end{itemize}
Based on these keywords, it was possible to formulate a search string: synonyms and their different spellings are connected using a Boolean \texttt{OR} operator and groups of synonyms are connected using an \texttt{AND} operator.
The initial search string was used to perform a tentative search;
it helped validate that the returned results contain relevant papers and helped refine the search string by adding more synonyms of selected keywords.
Additionally, the term \emph{user interface description language} and its initialism \emph{UIDL} are appended to the search string using \texttt{OR}s as a recognizable term that would not be matched by the previously described conjunction of keywords.
The final search string is shown in Listing~\ref{lst:general-search-string-rq-1}.
\begin{listing}
    \caption{The search string}
    \begin{minted}{text}
(
  (
    UI OR "user interface"
  )
  AND
  (
    abstract OR declarative OR intermediate OR
    universal OR polymorphic OR generic OR
    independent OR device-independent OR
    multi-platform OR multi-device
  )
  AND
  (
    description OR describing OR
    model OR modelling OR
    metamodel OR meta-model OR "meta model" OR
    language OR representation OR
    specification OR specifying
  )
)
OR
(
  "user interface description language" OR UIDL
)
    \end{minted}
    \label{lst:general-search-string-rq-1}
\end{listing}

To improve and narrow down the results of the searches, additional criteria are to be used:
\begin{itemize}
    \item papers should concern only the field of computer science
    \item papers should be written in English
    \item papers should be published in 2010 or later
\end{itemize}

The first two restrictions result from practical consideration and limit the amount of results to a reasonable one.
The date restriction is meant to narrow the focus only to the newest, most relevant contributions.

To find relevant papers, Scopus\footnote{\url{scopus.com}} and Web of Science\footnote{\url{webofscience.com/wos/}} will be used.
These engines are well-accepted in the scientific community and index research from most publishers (i.e.\ Springer, ACM, IEEE).
