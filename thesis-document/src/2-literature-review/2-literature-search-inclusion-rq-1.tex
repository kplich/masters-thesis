\subsection{Literature search and inclusion \textendash\ RQ 1}\label{subsec:literature-search-inclusion-rq-1}

After finishing the preparations, the next step is to identify papers that should provide a basis for the review.
The first step is to \textbf{perform the searches} in search engines (adjusting the search strings for each one).
Afterwards, the\ \textbf{search results} (search date, search areas, number of results, exact search string) \textbf{should be documented}.
If the original search results do not yield satisfying results, the search strings should be \textbf{refined as necessary}.

Simultaneously with the search, literature is going to be \textbf{tentatively screened for inclusion}, based on titles and abstracts of works.
This stage does away with formal selection criteria.
The papers are chosen for further consideration based on a brief intuitive judgement of suitability and relevance;
anything irrelevant to the discussion is going to be discarded during the second screening for inclusion.
In this stage, the decision is made after reading the paper's \textbf{introduction and conclusions} sections.
This time, the criteria for exclusion should be made \textbf{explicit and documented}.

After filtering out the least interesting texts, the last phase of the review aims to identify other articles
that should not be included in the review because of other reasons, such as insufficient quality or methodological flaws,
\textbf{based on their full content}.
Reasons for rejecting papers in this step also should be provided.

\todo[inline]{describe backwards and forward search for references!}

The above process also apply for the second research question.

\todo[inline]{make this a subsection for RQ1}

\subsubsection{Searches and initial screening}\label{subsubsec:searches-and-initial-screening}

% Please add the following required packages to your document preamble:
% \usepackage{booktabs}
% \usepackage{multirow}
\begin{table}[]
    \centering
    \caption{Results of the first stage of the literature review for RQ 1}
    \begin{tabular}{@{}rrcc@{}}
        \toprule
        \multicolumn{1}{l}{\textbf{Search engine}} & \multicolumn{1}{l}{\textbf{Search date}} & \multicolumn{1}{l}{\textbf{Search results}} & \multicolumn{1}{l}{\textbf{Papers qualified for further review}} \\
        \midrule
        Scopus                                     & 9.03.2022                                & 950                                         & 64                                                               \\
        IEEE                                       & 17.03.2022                               & 261                                         & 24                                                               \\
        ACM                                        & 20.03.2022                               & 178                                         & 18                                                               \\
        SpringerLink                               & 21.03.2022                               & 293                                         & 17                                                               \\
        \multirow{2}{*}{Semantic Scholar}          & \multirow{2}{*}{26.03.2022}              & 100 (36 600)                                & \multirow{2}{*}{28}                                              \\
                                                   &                                          & 100 (17 600)                                &                                                                  \\
        \multirow{2}{*}{DBLP}                      & \multirow{2}{*}{30.03.2022}              & 23                                          & \multirow{2}{*}{10}                                              \\
                                                   &                                          & 4                                           &                                                                  \\
        \midrule
        \multicolumn{3}{r}{\textbf{Total after exclusion of duplicates}}                                                                    & 119                                                              \\
        \bottomrule
    \end{tabular}
    \label{tab:results-first-stage-review-rq-1}
\end{table}

\todo[inline]{loose thoughts here}
\begin{itemize}
    \item other strings used in DBLP and Semantic Scholar - complex queries not supported;
    \item two queries used instead
    \item where possible, all works have been screened; exception semantic scholar -- only first 10 pages scanned
    \item total after excluding duplicates and unavailable PDFs
    \item little works describing the UIDLs directly -- instead hoping to see some descriptions mentioned incidentally
\end{itemize}
