\subsection{Discussion of results}\label{subsec:evaluation-discussion-of-results}

\subsubsection{Concepts missing}

\todo[inline]{revise this section once you get the actual results; right now it feels too much like too early conclusions}

Although the previous sections provide quite a comprehensive list of features, it is far from being complete.
It might be possible to describe a wide range of application in these simplest terms, more features are necessary to make full-fledged development of applications using these languages more feasible.
This section briefly describes the features that have not been found in any of the descriptions and which are to be found in various mobile, desktop and Web GUI technologies.

\paragraph{Specialized widgets}
As mentioned in the previous section, it is desirable for developers/modellers to work with a wide selection of components that neatly fit many possible use cases instead of developing them themselves.
Unfortunately, they might find the current state of the descriptions lacking, with only the most basic variants of components available in the descriptions.
For example, in some languages it is possible to use a grid layout;
however, none of the languages seems to have a masonry layout\footnote{Masonry layout displays content in columns, but unlike a grid, it does not require the conent to be aligned in the row axis, which makes the arrangement more balanced.} defined.

%\subparagraph{(Paged) documents}

%\paragraph{Support for touch-screens}

\paragraph{Dependency management}
As the languages are not mature and widespread, there is little incentive to create, disseminate and manage reusable pieces of UI description.
Therefore, there are not any mature mechanisms for managing any external dependencies or libraries comparable to what is available for conventional programming languages.

\paragraph{Code integration}
Currently, the languages presented are limited in their expressiveness by their limited set of actions that can be performed in response to events.
While it might be a welcome simplification for people unfamiliar with programming, it poses a grave limitation for developers.
It might be reasonable to accommodate for any gaps in the language by allowing to \enquote{patch} them with code instead.

% \subparagraph{Component lifecycle support}
% The languages could create a generic lifecycle and support it.

\paragraph{Global application state}
Most modern GUI programming enhances the event-driven paradigm with the principle of unidirectional data flow: events \enquote{flow} up (events from children components are handled by parent elements) and state \enquote{flows} down (parent components set the state of child components).
While the principle helps make the components reusable, testable and simpler to a certain extent, it fails in the case where data needs to be available throughout the application.
The response to this problem are \emph{state containers} that centralize the application state, making state management more predictable and flexible.
Future iterations of UIDLs could integrate these solutions to increase the flexibility and maintainability of developed applications.

\paragraph{Internationalization}
% \todo{chyba ze to jest w ktoryms z jezykow? idk}
Many production-grade applications require to be available in more than one language version.
However, to generate multiple versions of an application`, each using a different language would be impractical;
instead, any locale-dependent resources (primarily text) are stored outside application code and loaded during runtime, depending on the language set by the user/system.
Integrating such a mechanism directly into the language would increase the value and flexibility of languages.
% \todo[inline]{different text directions}

% \paragraph{Accessibility} \todo{not really sure that's a single feature that can be easily included}

\paragraph{Animations}
While it is possible to create full-scale applications without any animations, they can still make a big difference and transform a satisfactory user experience to an outstanding one.
Animations are useful for indicating navigation, interaction progress and state changes in a friendly and visually appealing way.
Without this feature in UIDLs, generated applications might miss out on this subtle aspect of user experience.

\paragraph{Design systems}
Design systems have emerged as a solution easing the development of various applications across multiple teams by establishing clear and common rules and style guides, defining reusable patterns and components.
By using them, organizations and companies can spend less time designing and implementing their applications, while achieving reliability and a unified appearance across all their products.
So far, there seems to be no explicit support for defining any parts of a style guide in the UIDLs analyzed which make them less suitable for large-scale development.

\paragraph{Dynamic appearance}
% todo: 4 sure?
Additionally, any support for making the appearance of some UI elements dependent on component/application state also seems to be largely absent.
The most important application for such a feature would be implementation of application themes (especially the famous dark theme).
This further prevents developers from implementing more engaging interactions and customizable experiences.
