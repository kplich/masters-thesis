\section{Introduction}\label{sec:introduction}

\todo[inline]{write as prose}

computing technology is more and more ubiquitous; it's not only for academics
or office workers - since the turn of the century, everyday people can also use
computers thanks to pretty and easy to use
\emph{graphical user interfaces} \todo{How did that happen, elaborate?}

designing an appealing GUI has never been an easy task for various reasons,
and it still takes a significant amount of time \todo[size=\tiny]{how much, exactly?
quote paper from \enquote{Evaluating user interface generation approaches
- model-based versus model-driven development} (first section)}
and is an important activity while designing software

there are many programs that can be used to design user interfaces;
nevertheless, despite years of technological advancements
both in the industry and in the academia\todo{proof?}, paper sketches are still
one of the most popular \todo{proof?} ways of creating UIs, especially during prototyping
or in the early stages of development

paper-based sketches are the preferred method for ideating on designs
because of their free-from nature that doesn't constrain creativity,
low barrier of entry, and ease of understanding;
unfortunately, the analogue medium is not a perfect one: sketches on paper
can only stay on paper and are difficult to manage \todo{prove benefits and disadvantages of paper-based sketches}

the sketches are useful for discussing UI ideas, but the results
always need to be translated into a digital form - either by a designer using a prototyping/design tool or by a developer and his development environment;
the process is rather tedious and mechanical \todo{proof?} and thus presents itself
as a candidate for (at least partial) automation;
time saved on translating a sketch into a prototype
could be then utilized more efficiently and fulfillingly

additionally, manual translation makes the resulting artifact not universal
(especially when developing directly in a programming language)
- choice of language/framework is usually an irreversible decision;
should a developer want to create a prototype in a different language
or design software, there's little possibility of translating the artifact
between those two and the manual work needs to be done again
\todo{how much of an issue is this?}

the goal of thesis is to address these problems and propose a method
of translating a paper-based sketch of a user interface
to an intermediate representation that can be converted into artifacts
more specific to certain technologies and software

\todo[inline]{complete the introduction with scope of the thesis and research questions}
