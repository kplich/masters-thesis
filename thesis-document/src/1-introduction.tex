\section{Introduction}\label{sec:introduction}

\subsection{Overview of model-based user interface development}\label{subsec:user-interfaces-are-important}
Computer technologies are now ubiquitous thanks to user interfaces.
They enable billions of people to comfortably use computers, phones and many other devices.
Nowadays, a usable and friendly interface can decide about a success or failure of a product~\cite{Offutt2002}.
No wonder then, that design, implementation and maintenance of UI are a central point of developing software products~\cite{Anderson2010}.
Research shows that these processes take up roughly the half of time devoted to developing the whole product~\cite{Myers1992}.
% All this research takes place as a part of the rich discipline of human-computer interaction ^[*The psychology of human-computer interaction*, 1983, Card, Moran, Newell].

% \subsection{Diversity and complexity of UIs causes difficulties}\label{subsec:diversity-and-complexity-of-uis-causes-difficulties}
Development of UIs has never been easy, mostly because of technical issues~\cite{Six1991}.
Over time, though, the challenge has taken on a new dimension.
The number, diversity and connectedness of devices on the market has risen exponentially~\cite{Cisco2020}.
Such an explosion of complexity has made it ever so harder to deliver a consistent and satisfying user experience to users.
Together with maturation of the computing technologies, approaches and methods for systematic development of UIs have also been devised in order to simplify and speed up the process while also reducing costs and mistakes.

% \subsubsection{UIMSs \& MBUID}
The process has started with user interface management systems – tools for \enquote{development and management of the interaction in an application domain across varying devices, interaction techniques and styles}~\cite{Betts1987}.
However, because of their shortcomings, a new paradigm under the name of model-based user interface development (MBUID) has emerged, where a model is \enquote{a central description of all aspects of an interface design}~\cite{Puerta1994}.
Since then, there have been many achievements in the area of model-based UI development – the approaches progressed from simply generating a UI based on a model to managing the diversity and complexity of UIs, handling different platforms, devices, interaction modalities, user states and usage contexts~\cite{Meixner2011}.
As the maturity of solutions increases, it is expected that model-based approaches will become model-driven, i.e., fully automated and integrated with other parts of the development process in the spirit of model-driven development~\cite{Ruiz2018}.

% \subsubsection{No-code \textendash\ no-wave?}
A paradigm similar to model-driven development has emerged in recent years from the business side of systems development in the form of \emph{no-code} or \emph{low-code development}~\cite{Rymer2019}.
Although considered as an exercise in rebranding of MDD and not particularly groundbreaking in technical terms, the intensive business support may serve as a vehicle for promoting concepts aligned with MDD~\cite{Bock2021}.

\subsection{The Cameleon Reference Framework}\label{subsec:the-cameleon-reference-framework}

The Cameleon Reference Framework (CRF)~\cite{calvary_cameleon_framework_2002, calvary_cameleon_glossary_2002} represents an important development in the field of model-based UI development.
It defines many aspects of the area and organizes them into a framework for evaluation of tools and approaches related to the discipline.
The most important contribution of the project is the definition of four levels of abstraction in UI development:
\begin{samepage}
\begin{itemize}
    \item \textbf{tasks and concepts (T+C)}: on this level, the UI is viewed only from a \textbf{functional and conceptual perspective};
    practically no thought is given to the implementation of the interaction
    \item \textbf{abstract UI (AUI)}: on this level, the UI is structured \textbf{independently from modality of interaction}\,\textendash\,it is not yet known if it will be a graphical, textual, or voice UI (to name a few examples)
    \item \textbf{concrete UI (CUI)}: on this level, the modality of the UI is known and the specification is only \textbf{independent from platform};
    e.g.\ the definition assumes a graphical UI and specifies a text input\,\textendash\,however, it's not yet decided, what technology will be used to realize it
    \item \textbf{final UI (FUI)}: on this level, the executable UI, written in a \textbf{particular technology} is generated
\end{itemize}
\end{samepage}
Models at these stages can be related to one another through transformations: \emph{abstraction}, \emph{reification} and \emph{translation}.
For example, the AUI stage can be \emph{abstracted} from the CUI stage\,\textendash\,the modality-specific information is lost (can't be expressed) at this stage;
conversely, the AUI stage is a result of  \emph{reification} of the T+C stage by adding more information about how the UI should be structured\,\textendash\,the two transformations can be though of as inverses of one another.
Step-by-step reification of models is a primary way of developing a user interface on the basis of models.
Abstraction can be used to reverse engineer models from an existing UI.
These two transformations are complemented by \emph{translation}\,\textendash\,a transformation between models at the same stage of abstraction (e.g.\ translation of a graphical CUI model to a voice CUI model.)

\subsection{Expressiveness as an important model quality in MBUID}\label{subsec:model-quality-in-mbuid}
Naturally, in order to be useful for the goals of model-based UI development, the models need to satisfy certain properties.
Hailpern and Tarr discuss this problem in the context of model-driven development in general: they describe UML as an example of a model (or a set of models) that might not satisfy the needs of approaches in the domain~\cite{Hailpern2006}.
In that case, authors point to complexity and insufficient semantics as language deficits that limit its expressive capacity, potential for automation and thus discourage its wider adoption.

Vanderdonckt also identifies model quality as one of the challenges in the field of MBUID~\cite{Vanderdonckt2008}: semantics, as well as syntactic and stylistic rigor, are mentioned as important features of a model.
These features help ensure that the model satisfies desirable properties, such as completeness, consistency, correction, expressiveness, and conciseness.
In the paper, expressiveness is defined as \enquote{[the] ability of a model to express via an abstraction \textbf{any} [real-world] \textbf{aspect of interest}}.
Flexibility of modelling is indeed important, as it helps better satisfy application requirements;
if not provided, developers are forced to modify generated code manually~\cite{Pederiva2007}.
As suggested by Aquino et al.~\cite{Aquino2010}, this could be realized by adding another model to the approach.
However, such an approach stands in opposition to another challenge of MBUID: the risk of proliferation of models necessary to fully represent the final UI~\cite{Vanderdonckt2008}.
At best, the UI model should allow flexibility on its own, without any additional constructs.

\subsection{Goal and scope of the thesis}\label{subsec:goal-and-scope}

Various reviews show that UI representations are for the most part not yet flexible enough for most applications~\cite{Ruiz2018, Souchon2003}.
However, they do not provide a detailed account of how they evaluate expressiveness of UI models;
there also does not exist any analogue of Cameleon Framework that could help assess representations in this regard.
The goal of this thesis is therefore to more thoroughly investigate this aspect of UI representations and report the results.
Expressiveness is understood as a model's capability to capture information relevant to generation of the final user interface.

To realize the goal, a systematic literature review will be conducted.
Systematic literature reviews are a widely accepted way of \enquote{identifying, evaluating and interpreting all research available and relevant to a specific research question}~\cite{kitchenham_guidelines_2007}.
The process of planning and executing the review is described in section~\ref{sec:literature-review}.
\todo[inline]{kolejne sekcje beda tu dalej opisane}
