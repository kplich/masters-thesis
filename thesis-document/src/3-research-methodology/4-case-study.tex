\subsection{Case study}\label{subsec:case-study}
This section describes the practical experiment used to evaluate representations.

\subsubsection{Scenario and context}
The goal of this experiment was to evaluate the expressiveness of representations in a practical context.
This was achieved by recreating a few screens of a fictional application.

The application designed for this experiment was modeled after Trello\furl{trello.com} \textendash\ a well-known Web application for creating Kanban-style lists.
The choice was motivated by the application's popularity and relative complexity.
The original app is, naturally, very complex \textendash\ its capabilities include detailed management of workspaces, users, boards, as well as editing and managing content-rich cards.
To keep the experiment manageable, it would only concern minimal management of cards in a board.
This is a minimal, yet representative set of use cases expected from an application (it covers all four CRUD operations).
Additionally, the chosen set of functionalities covers most of the concepts defined in the previous section.


The evaluation will be carried out by attempting to recreate a certain existing UI in the languages.
Because of the lack of availability, the assessment will be carried out only for certain languages.

This section describes the test case against which the languages will be evaluated.
