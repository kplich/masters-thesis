\subsection{Basis for evaluation}\label{subsec:basis-for-evaluation}

This section presents concepts related to programming user interfaces included in the selected papers.
Their definitions were formulated by the author through relating them with concepts found in actual GUI technologies.
They cover three areas described in following subsections: architecture and basic elements, logic and behavior, and appearance.

\subsubsection{Architecture and basic elements of UI}
This category encompasses concepts that enable modellers to describe and organize the structure of the developed application.

\paragraph{Hierarchical description of structure}
Most presented representations describe the structure and contents of GUIs using hierarchical/tree-like structures, where each element can contain multiple \emph{children} elements and only one \emph{parent} element.
The choice reflects the structure of descriptions used by many final UI technologies (HTML\furl{https://developer.mozilla.org/en-US/docs/Learn/HTML/Introduction_to_HTML/Getting_started\#nesting_elements}, Android XML\furl{https://developer.android.com/guide/topics/resources/layout-resource}, XAML\furl{https://learn.microsoft.com/en-us/dotnet/desktop/wpf/xaml/}, FXML\furl{https://docs.oracle.com/javase/8/javafx/api/javafx/fxml/doc-files/introduction_to_fxml.html})
The root node of such a structure is often highlighted in the descriptions and named an \emph{application node} or a \emph{project node}.

\paragraph{View}
An application consists of \emph{views}/\emph{screens} \textendash\ collections of controls that are used together in order to fulfill a certain use case.
The closest real-world analogue would be an Android \texttt{Activity}\furl{https://developer.android.com/reference/android/app/Activity}, representing \enquote{a single, focused thing that the user can do.}
A special kind of view is a \emph{dialog (window)}.
It is displayed outside the main flow of the GUI (mostly above) that can be used to display important information to a user (or receive confirmations, etc.).
The concept is realized e.g.\ in the \texttt{<dialog>} element\furl{https://developer.mozilla.org/en-US/docs/Web/HTML/Element/dialog} defined in the HTML language.

\paragraph{Component (element)}
The content of UIs is described with \emph{elements}/\emph{components} \textendash\ they are the smallest parts of a description of a view.
The category of components is broad;
it is possible to look at and classify them from multiple perspectives.

\subparagraph{Containers vs widgets}
This classification describes if the element can contain and handle children components \textendash\ in that case, it is called a \emph{container}.
Otherwise, when it is meant to be used on its own, it is called a \emph{widget}.
Most UI elements can be classified as widgets, e.g.\ buttons, form fields, multimedia, etc.
However, container components have an important role in the UI, grouping related elements together and providing a usable structure to the interface.
Examples of container components include lists, cards\furl{https://material.angular.io/components/card/overview}, and layout components (e.g.\ grid\furl{https://mui.com/material-ui/react-grid/} or relative\furl{https://learn.microsoft.com/en-us/windows/apps/design/layout/layout-panels\#relativepanel}).
The classification does not have to be binary \textendash\ for example, a tabbed layout\furl{https://developer.apple.com/documentation/appkit/nstabview\#2555818} contains other content, but also has its own controls responsible for switching tabs.

\subparagraph{Output vs input components}
This classification describes whether components enable interactivity of the GUI (can be used to provide data or change application state).
\emph{Output} components are only used to display information and do not let the user change application state in any way.
Examples of output components could be text elements, labels, or multimedia components.
On the other hand, \emph{input} components allow user to interact with the application, e.g.\ by providing data using input fields or triggering any changes in application state.

\subparagraph{Predefined vs custom vs external components}
A component may be \emph{predefined} by the authors directly in the core description.
Although, from the developer's perspective, it is desirable to have a wide range of predefined components, it is often difficult and impractical for the creators of the representation to enumerate them all.
Because of this, they often decide to define the most useful and popular widgets and incorporate an extension mechanism into their language.
One such mechanism is to allow for defining \emph{custom} components within the description.
Custom components are defined using features of the language and other components (be it predefined, other custom or external components).
Another example of extending the description is using \emph{external} components.
As the name suggests, they are defined outside the description \textendash\ they may come from e.g.\ existing component libraries.

The fitting analogy is the relationship between HTML, Web Components\furl{https://developer.mozilla.org/en-US/docs/Web/API/Web_Components} and client-side JavaScript frameworks\furl{https://developer.mozilla.org/en-US/docs/Learn/Tools_and_testing/Client-side_JavaScript_frameworks/Introduction}.
HTML defines a set of \emph{predefined} elements useful for describing documents.
Client-side frameworks provide mechanisms to use these elements to create \emph{custom} components that fulfill application-specific needs.
However, to speed up development time, both HTML and frameworks may want to reuse \emph{external} components developed by third parties \textendash\ Web Components allow developers to use them directly in HTML\@.

\paragraph{Asset (Resource)}
\emph{Assets} or \emph{resources} are files that are used by the application for various purposes.
As an example, Android applications can contain resources that can be used in different contexts, e.g.\ fonts, configuration files and binary files\furl{https://developer.android.com/guide/topics/resources/providing-resources}.

\paragraph{Modularization}
\emph{Modularization} encapsulates the possibility of splitting the description of a particular UI into smaller, possibly reusable parts (preferably across multiple files) that may be imported.
Because of its importance for creating uniform and maintainable applications, this capability is a strict necessity in standard programming languages;
however, not all UIDLs provide such a possibility.

\subsubsection{Logic and behavior of UI}

This category comprises concepts that allow modellers to make the UI interactive.

\paragraph{Events, handlers, and actions}
As the \emph{event}-driven paradigm is the standard among GUI applications~\cite{wang2016event}, it is not surprising that practically all the description languages analyzed describe the logic of user interfaces in similar terms.
In short, user interaction with a certain element triggers events that cause \emph{event handlers} attached to the element to be called.

In most GUI technologies, the handlers are implemented using callbacks or plain methods, registered to be called each time an event is emitted.
In final technologies, handlers can execute practically any code.
However, modelling fully-fledged programming would be out of scope for these representations.
For this reason, they usually only specify a set of GUI-related, predefined \emph{actions} that can be executed and combined, e.g.\ navigation to a different screen or modification of internal state of a component;
Sometimes the ability to integrate with external logic (e.g.\ REST backends) is provided through a restricted mechanism.
Only rarely do the representations abandon this approach and provide the possibility to complement the abstract description directly with a more expressive programming language.

Description languages (similarly to final technologies) often predefine certain events, applicable to most elements \textendash\ the most popular are mouse events (e.g.\ a \emph{left-}/\emph{right click}) and keyboard events (e.g.\ pressing a key or a set of keys), although many other types of events could exist\furl{https://developer.mozilla.org/en-US/docs/Web/Events}.
Additionally, some description languages also allow defining custom events, which is especially useful with custom widgets.

\paragraph{Control structures}
Control structures such as \texttt{if-else} blocks, \texttt{while} loops, and \texttt{for} loops are an indispensable part of every programming language.
These mechanisms might also be included in the UIDL to increase the expressive power of event handlers.
Moreover, logic is also useful while defining the content of views/components: \emph{conditional rendering} allows for displaying components based on a condition while \emph{iteration} is used to display components multiple times on the basis of a collection.

A special kind of control and conditional execution, prominently featured in many UIs, is (client-side) \emph{form validation}\furl{https://developer.mozilla.org/en-US/docs/Learn/Forms/Form_validation\#what_is_form_validation}.
As forms are a very popular idiom of interaction in UIs, technologies often provide dedicated APIs that relieve developers from implementing the functionality from scratch;
therefore a similar solution would also be expected in a expressive UIDL\@.

\paragraph{Expressions}
To enable any reasonable logic and processing of data, expressions that operate on various data types are necessary.
Just as regular programming languages, UIDLs can provide the possibility to reference data from different sources and use them in some arithmetic, textual or boolean expressions.
These expressions can be used both within handlers and within the content of the UI;
the mechanism of replacing the variables or expressions in a string/template with their values is known in some UI technologies as \emph{interpolation}\furl{https://angular.io/guide/interpolation}.

\paragraph{State}
It is necessary to define, read and modify some sort of state (e.g.\ loaded user data, currently viewed item, form data) to be able to encapsulate logic in components or views that make up the UI\@.
In final UI technologies (as well as most software in general), this is realized simply by means of declaring \emph{variables}, \emph{fields} or \emph{properties}.

\paragraph{Custom component definition}
Some approaches also allow defining own, reusable stateful components, just as many GUI languages do\furl{https://docs.oracle.com/javafx/2/fxml_get_started/custom_control.htm}\furl{https://vuejs.org/guide/essentials/component-basics.html} \textendash\ this makes the code or models more maintainable, which helps reduce development time and costs.
This is made possible thanks to techniques described in other parts of this section: content, handlers, custom events, internal state, appearance.

An additional aspect that needs to be considered when defining custom components is the mechanism of input parameters \textendash\ they are necessary to make components manageable, reusable and useful in different contexts.
One type of input parameters allows for configuring the components and changing their appearance or functionality \textendash\ these parameters can be called \emph{attributes}.
In many XML-based languages, properties are passed as attributes\furl{https://developer.mozilla.org/en-US/docs/Learn/HTML/Introduction_to_HTML/Getting_started\#attributes}\furl{https://learn.microsoft.com/en-us/dotnet/desktop/wpf/xaml/?view=netdesktop-7.0\#attribute-syntax-properties}.
Client-side Web frameworks additionally use an additional mechanism (alongside HTML attributes) for passing data to components, called \emph{props}\furl{https://vuejs.org/guide/essentials/component-basics.html\#passing-props} (short for \enquote{properties}).
The other type of input allows for defining the content of components.
In XML-based languages, for example, this typically happens by simply nesting elements\furl{https://developer.mozilla.org/en-US/docs/Learn/HTML/Introduction_to_HTML/Getting_started\#nesting_elements}.

\subsubsection{Appearance}
This category describes concepts that give developers influence over the appearance of the GUI\@.

\paragraph{Layout}
\emph{Layout} defines how children are laid out within a container, e.g.\ a border layout, where each element must be aligned to one of the window borders or the center of the window, or a relative layout, where elements can be aligned to other elements or the parent container.
Additionally, the layout can be further customized with properties such as separation between elements, element alignment, etc.

Usually, layout is implemented through a dedicated container component that is responsible for placing its children\furl{https://learn.microsoft.com/en-us/windows/apps/design/layout/layout-panels}.
However, in the Web stack, layout is defined in the presentation layer using CSS properties\furl{https://developer.mozilla.org/en-US/docs/Web/CSS/display}.

Additionally, it's a good practice nowadays to make interface layouts accommodate displays of different sizes and fit accordingly \textendash\ it is called \emph{resposive design}.
Some descriptions make information about the display available to the view or component which allows them to better adapt to different devices.
The main example is usage of media queries in CSS\footnote{\url{https://developer.mozilla.org/en-US/docs/Web/CSS/CSS_media_queries/Using_media_queries}} to change the appearance of content based on device characteristics (e.g.\ resolution or screen width).

\paragraph{Sizing}
\emph{Sizing} allows the component to define how much space of its parent it needs to take.
There usually seem to be at least two options available: fixed size, defined in some absolute units, such as points or pixels, relative size, mostly defined as percentages.
An example of a different sizing option would be proportional sizing (where all elements indicate the proportion of the container they need to take up).

\paragraph{Positioning}
\emph{Positioning} is a complementary facet of sizing.
Part of this topic overlaps with layout \textendash\ it also concerns how items are positioned.
This can happen however in two ways \textendash\ either a component tells its children how they should be laid out or the child can override or modify this setting.
Apart from the default setting, a child component can specify its position as relative to the default position or position itself absolutely.

\paragraph{Other presentation properties}
Most approaches allow for most basic styling: padding (internal separation), margins (external separation), shadow, borders, backgrounds.
Additionally, there might be support for text styling, such as font type, size, weight, decoration, etc.
Finally, some descriptions make it possible to specify the appearance of portions of applications in probably most comprehensive and robust way \textendash\ by using CSS directly.
