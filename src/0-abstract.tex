\pdfbookmark[0]{Abstract}{streszczenie.1}
\begin{abstract}
UI development is a costly process due to the diversity of existing devices and technologies.
Various representations used to model user interfaces and manage their variety differ in the level of detail they can address.
This thesis aims to compare UI representations concerning their expressiveness -- their ability to capture details relevant to the process of generation of code.

Relevant UI representations were identified through a systematic literature review and analyzed to summarize concepts related to UI programming.
To assess the representations, a set of criteria and a case study were developed, based on concepts described in the analysis.

The results indicate that most UI representations cannot be considered expressive, which limits their applicability in UI development.
The most expressive representation, OpenUIDL, scores around 70\% -- it has sound architectural fundamentals and rich capabilities for describing appearance but lacks meaningful support for managing behavior.

The thesis points out the gap between the current and desired capabilities of UI representations.
The research presented can be expanded upon in multiple ways, e.g., by replicating it with other notations or refining the evaluation criteria.
\end{abstract}
\mykeywords

\selectlanguage{polish}
\begin{abstract}
Rozwój interfejsu użytkownika jest kosztownym procesem ze względu na różnorodność istniejących urządzeń i technologii.
Różne reprezentacje używane do modelowania interfejsów użytkownika i zarządzania ich różnorodnością różnią się poziomem szczegółowości, które mogą uwzględnić.
Niniejsza praca ma na celu porównanie reprezentacji interfejsu użytkownika pod kątem ich ekspresywności -- zdolności do uchwycenia szczegółów istotnych dla procesu generowania kodu.

Istotne reprezentacje interfejsu użytkownika zostały zidentyfikowane poprzez systematyczny przegląd literatury i przeanalizowane w celu podsumowania koncepcji związanych z programowaniem interfejsów użytkownika.
Aby ocenić reprezentacje, opracowano zestaw kryteriów i studium przypadku na podstawie koncepcji opisanych w analizie.

Wyniki wskazują, że większości reprezentacji interfejsu użytkownika nie można uznać za ekspresyjne, co ogranicza ich zastosowanie w rozwoju interfejsów użytkownika.
Najbardziej ekspresyjna reprezentacja, OpenUIDL, uzyskała wynik około 70\% -- ma solidne podstawy architektoniczne i bogate możliwości opisywania wyglądu, ale brakuje jej znaczącego wsparcia dla zarządzania zachowaniem.

Praca wskazuje na lukę między obecnymi i pożądanymi możliwościami reprezentacji interfejsów użytkownika.
Przedstawione badania można rozszerzyć na wiele sposobów, na przykład poprzez powtórzenie ich przy użyciu innych notacji lub udoskonalenie kryteriów oceny.
\end{abstract}
\mykeywords
\selectlanguage{english}
