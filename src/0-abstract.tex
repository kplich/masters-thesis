\pdfbookmark[0]{Abstract}{streszczenie.1}
\begin{abstract}
UI development is a costly process, mainly due to the diversity of existing devices and technologies.
Various models used to manage this variety differ in the scope and level of detail they can address.
This thesis aims to compare UI representations concerning their expressiveness \textendash\ their ability to capture details relevant to the process of generation of code.

Relevant UI representations were identified through a systematic literature review.
Results were analyzed to summarize concepts related to UI programming which they address.
To assess the representations, a set of criteria and a case study were developed, based on concepts described in the analysis.
The results of the evaluation indicate that most UI representations cannot be considered expressive, which severely limits their applicability in UI development.
The most expressive representation, OpenUIDL scores around 70\% \textendash\ it has sound architectural fundamentals and rich capabilities for describing appearance but lacks meaningful support for managing behavior.

The thesis points out the gap between the current and desired capabilities of UI representations that should be addressed in existing and future representations.
The research presented can be expanded upon in multiple ways, e.g., by refining the description of the domain of UI programming or by replicating it with other notations.
\end{abstract}
\mykeywords

\selectlanguage{polish}
\begin{abstract}
Rozwój interfejsu użytkownika jest kosztownym procesem, głównie ze względu na różnorodność istniejących urządzeń i technologii.
Różne modele używane do zarządzania tą różnorodnością różnią się zakresem i poziomem szczegółowości, które mogą uwzględnić.
Niniejsza praca ma na celu porównanie reprezentacji interfejsu użytkownika pod kątem ich ekspresywności i zdolności do uchwycenia szczegółów istotnych dla procesu generowania kodu.

Istotne reprezentacje interfejsu użytkownika zostały zidentyfikowane poprzez systematyczny przegląd literatury.
Wyniki przeanalizowano w celu podsumowania koncepcji związanych z programowaniem interfejsu użytkownika, do których się odnoszą.
Aby ocenić reprezentacje, opracowano zestaw kryteriów i studium przypadku na podstawie koncepcji opisanych w analizie.
Wyniki oceny wskazują, że większości reprezentacji interfejsu użytkownika nie można uznać za ekspresyjne, co poważnie ogranicza ich zastosowanie w rozwoju interfejsów użytkownika.
Najbardziej ekspresyjna reprezentacja, OpenUIDL, uzyskała wynik około 70\% \textendash\ ma solidne podstawy architektoniczne i bogate możliwości opisywania wyglądu, ale brakuje jej znaczącego wsparcia dla zarządzania zachowaniem.

Praca wskazuje na lukę między obecnymi i pożądanymi możliwościami reprezentacji interfejsów użytkownika, które powinny zostać uwzględnione w istniejących i przyszłych reprezentacjach.
Przedstawione badania można rozszerzyć na wiele sposobów, na przykład poprzez udoskonalenie opisu dziedziny programowania interfejsów użytkownika lub powtórzenie ich przy użyciu innych notacji.
\end{abstract}
\mykeywords
\selectlanguage{english}
