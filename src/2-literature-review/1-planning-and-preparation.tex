\section[Planning and preparation]{Planning and preparation for the review}\label{sec:planning-and-preparation-for-review}

\subsection{Defining research questions and justifying the review}\label{subsec:defining-research-questions-and-justifying-the-review}

There have been several surveys of UIDLs conducted in the past~\cite{Souchon2003, guerrero_garcia_theoretical_2009, guerrero_garcia_theoretical_2011, Jovanovic2013}.
They mention the expressiveness requirement directly or indirectly (number of tags, coverage of concepts) in their comparisons;
however, they are not systematic and well-documented.
The latest review appeared in 2013, and no recent overviews seem to exist.
Additionally, the publications focused on XML-based languages and did not research any other forms of representing user interfaces.

A systematic review by Ruiz et al~\cite{Ruiz2018} defines ten criteria for evaluating MBUID environments.
Two of them (code generation and design flexibility) are relevant to the purposes of this thesis.
However, the paper only provides a general, nonspecific methodology for evaluation.
Additionally, it does not focus on representations themselves and does not include approaches from later than 2015.

\subsection{Defining search criteria}\label{subsec:defining-search-criteria}
First, keywords and synonyms relevant to identifying existing UI representations were identified:
\begin{itemize}
    \item \textbf{terms related no non-direct descriptions}: abstract/universal/generic/declarative/\\intermediate/polymorphic/(platform-)independent/(device-)independent/\\multi-platform/multi-device
    \item \textbf{terms related to user interfaces}: user interface/UI
    \item \textbf{synonyms of \enquote{representation}}: representation/(meta-)model/language/\\specification/description
\end{itemize}
Based on these keywords, it was possible to formulate a search string: synonyms and their different spellings were connected using a Boolean \texttt{OR} operator, and groups of synonyms -- using an \texttt{AND} operator.
The initial search string was used to perform a tentative search;
it helped validate that the returned results contained relevant papers and helped refine the search string by adding more synonyms of selected keywords.
Additionally, the term \emph{user interface description language} and its initialism \emph{UIDL} were appended to the search string using \texttt{OR}s as a recognizable term that would not be matched by the previously described conjunction of keywords.
Listing~\ref{lst:general-search-string-rq-1} shows the final search string.
\begin{lstlisting}[label=lst:general-search-string-rq-1,caption=The search string, basicstyle=\ttfamily]
(
  (UI OR "user interface")
  AND
  (
    abstract OR declarative OR intermediate OR universal OR polymorphic OR
    generic OR independent OR device-independent OR multi-platform OR multi-device
  )
  AND
  (
    description OR describing OR model OR modelling OR
    metamodel OR meta-model OR "meta model" OR
    language OR representation OR specification OR specifying
  )
)
OR ("user interface description language" OR UIDL)
    \label
\end{lstlisting}

To improve and narrow down the results of the searches, additional criteria were used:
\begin{samepage}
\begin{itemize}
    \item papers should concern only the field of computer science
    \item papers should be written in English
    \item papers should be published in 2010 or later
\end{itemize}
\end{samepage}

The first two restrictions resulted from practical consideration and limited the number of results to a reasonable one.
The date restriction was intended to narrow the focus to the newest, most relevant contributions.

To find relevant papers, Scopus\furl{scopus.com} and Web of Science\furl{webofscience.com/wos/} were used.
These engines are well-accepted in the scientific community and index research from most publishers (e.g., Springer, ACM, IEEE).
