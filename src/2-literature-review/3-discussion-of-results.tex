\section{Discussion of results}\label{sec:review-discussion-of-results}

This section provides the answer to RQ1: \enquote{What abstract UI representations exist?}

\subsection{Honorable mentions}\label{subsec:honorable-mentions}
To give a better overview of the domain, the section describes documents considered during the search but, in the end, were not included in further research.
The CUI model from CIAT-GUI~\cite{Molina2012-my}, QuiXML from MoCaDiX~\cite{Vanderdonckt2019-av}, and omniscript~\cite{Ulusoy2019-jh} all look like promising solutions to the problem of the thesis.
Unfortunately, the papers neither define the concrete UI representation in sufficient detail nor provide additional sources.

The other category of literature not considered in this study is the collection of established UIDLs, e.g., IFML~\cite{Brambilla2014-ln}, UIML~\cite{Abrams1999}, or UsiXML~\cite{Limbourg2005}.
Although they are well-regarded in the literature and often appeared in papers reviewed as the language of choice for modeling UIs, they were not considered further.
IFML and UIML were left out, as they expect the more concrete description to be specified in a different model.
For example, UIML provides an extension mechanism of vocabularies that can define the mapping of abstract components to implementation-specific classes or elements.
On the other hand, UsiXML also encompasses a concrete description.
Unfortunately, the language is not widely available nowadays;
the website of the UsiXML project does not contain any practical resources or specifications~\furl{http://www.usixml.org/en/software.html?IDC=239}.

The last category of excluded papers encompasses solutions for generating UI in other modalities.
Examples include descriptions of UI in virtual reality~\cite{Olmedo2015} or voice UIs~\cite{steinberger2020domain}.

\subsection{Selected papers}\label{subsec:selected-papers}

The 17 selected papers can be split into a few topical categories: UIDLs or DSLs, MBUID approaches, UI migration approaches, and reviews of GUI descriptions.

\subsubsection{UIDLs and DSLs}
5 languages described in 7 documents were found:

\paragraph{MARIA (2009)}
MARIA~\cite{Paterno2009, MariaPDF} (abbreviated as \enquote{Model-based lAnguage foR Interactive Applications}) is a language developed by the National Research Council of Italy on the basis of another UIDL, TERESA XML~\cite{mori2004}.
Its distinguishing feature is support for applications based on Web services described using WSDL\@.
The language has also been submitted for standardization at the W3C~\footnote{\url{https://www.w3.org/2005/Incubator/model-based-ui/XGR-mbui-20100504/}, last accessed 2023-03-24};
however, the activity does not seem to have concluded consequentially.

\paragraph{XANUI (2016)}
XANUI~\cite{hermida2016xanui} positions itself as a DSL for developing rich Internet applications (RIAs) and mobile applications in combination with the OOH4RIA methodology~\cite{Meli2008}.
What differentiates it from other contributions in the domain is that it is annotation-driven and can thus be integrated with existing development artifacts (especially XML files).

\paragraph{HCIDL (2018)}
HCIDL~\cite{Gaouar2018} (Human-Computer Interface Description Language) aims to enable the modeling of \enquote{multi-target, multimodal, and plastic} user interfaces.
Its primary focus, however, seems to be on the modeling capabilities of mobile devices and their sensors.

\paragraph{Quid (2018-2019)}
Quid~\cite{molina2018quid, Molina2019} is an online tool\furl{https://quid.metadev.pro/} for prototyping Web Components in the browser with its own minimal DSL, developed by a Spanish company specializing in building DSLs on the Web\furl{https://metadev.pro/}.
It allows generating components for multiple Web frameworks and visualizing them in real time.

\paragraph{OpenUIDL (2020)}
OpenUIDL~\cite{Moldovan2020} is another UIDL released by a company\furl{https://teleporthq.io/} and shipped with an online tool~\footnote{\url{https://play.teleporthq.io/signin} (as of March 21$^{st}$, 2023 the tool requires sign-up)}.
Although the publication claims to be able to support \enquote{omni-channel UIs} on multiple platforms, its focus seems to lie mostly on generating RIAs.

\subsubsection{MBUID approaches}

The review found six papers proposing MBUID approaches presenting their own UI model.

Bouchelligua et al.\ (2010)~\cite{Bouchelligua2010} present an approach for developing plastic UIs (that adapt to the context of use) for information systems.
In particular, the paper describes the usage of BPMN for task modeling tasks in a model-driven process.
The approach is fully compliant with the CRF\@.

Kryštof (2010)~\cite{kryvstof2010lpgm} developed a UML profile for the platform-independent development of UIs for desktop and web applications.

Achilleos et al.\ (2011)~\cite{Achilleos2011} focused on developing \enquote{device-aware} Web services.
The paper specifies a Presentation Modelling Language used to describe the graphical interface of the client.
The code for communication with the service backends is meant to be generated by other methods based on their WSDL descriptions.

Miao et al.\ (2017)~\cite{Miao2017} describe a model-based method of transforming the platform-independent model of the UI into a platform-specific one.
The authors present the metamodel as a UML class diagram.

Khan et al.\ (2021)~\cite{Khan2021} present a UML profile for mobile UIs and a transformation tool that transforms model instances into React Native code.
Their approach also considers presentation details such as element styling and layout.

Soude and Koussonda (2022)~\cite{Soude2022} propose a modeling language as a part of a model-driven approach for unifying the development of user interfaces.
Their method mostly primarily deals with generating web applications in multiple web frameworks.

\subsubsection{Other publications}

This section summarizes the rest of the documents included in the review.

Verhaeghe et al.\ (2021)~\cite{Verhaeghe2021visual, Verhaeghe2021behavior} have been working on platform-independent migration of GUIs.
The initial motivation was a large-scale migration of multiple GWT apps to Angular frontends;
however, the authors point out that their approach could also support future migrations to other technologies.
It accomplishes the goal by establishing a pivot GUI metamodel that represents the original application and serves as a basis for generating a new one.

Chmielewski et al.\ (2016)~\cite{Chmielewski2016} analyzed eight popular UIDLs and declarative UI descriptions in the context of the development of mobile UIs.
The criteria for judgment were expressiveness, clarity, support for event handlers, and support for the internationalization of UIs.
According to the review, no language allowed for implementing all experimental UIs, indicating the potential for further development in the area.

\subsubsection{Final remarks}

The selected papers come from the past decade and beyond, with contributions dating back to 2009 (MARIA) and as recent as 2022 (Soude and Koussonda).
They also approach the problem from multiple angles: the works include self-contained UIDLs and models designed for a particular method.
The number of new representations found further justifies the need for the review -- many of them were not previously analyzed in the literature.
Similarly, the existence of only one paper describing an evaluation of the expressiveness of UI representations validates the goal of the thesis.
