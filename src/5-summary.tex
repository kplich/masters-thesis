\chapter{Summary}\label{ch:summary}

The goal of the thesis was to evaluate the expressiveness of abstract UI representations, understood as the ability to influence the implementation details of the generated interface (final UI in Cameleon Reference Framework).
In order to gather information for the research, a systematic literature review was conducted.
The documents found in the process were analyzed to find out what features they implement.
The survey identified and detailed three major areas of UI development: architecture, behavior, and appearance; it was then used to define the criteria for evaluation of representations were defined.
To complement the theoretical assessment, a case study and a list of requirements were also developed.

The analysis of the results of the study shows that in general, the representations' expressiveness ranges from low to moderate -- they scored around 30--40\% in the theoretical evaluation.
Some representations achieved considerable scores in individual areas of evaluation, but none of them were adequately functional across all areas, which limits their applicability in general development.
This is true even of the most expressive representation found -- OpenUIDL; although its capabilities could suffice for prototyping, they will probably not meet the needs of developers.

The findings of the study indicate the need for further progress in user interface representations;
even though the gap between the current and desired capabilities of UI representations is narrow and possible to bridge, it remains unfilled.
New or existing representations can use the study as a point of reference for development.
Other directions of future research include revisions of the list of presented criteria and replicating the study with a broader set of representations.
