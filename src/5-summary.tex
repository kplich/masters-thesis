\chapter{Summary}\label{ch:summary}

The goal of the thesis was to evaluate the expressiveness of abstract UI representations, understood as the ability to influence the implementation details of the generated interface (final UI in the Cameleon Reference Framework).
A systematic literature review was conducted to gather information for the research.
Analysis of the documents found determined features implemented by the described representations, grouped into three areas of UI development: architecture, behavior, and appearance.
The survey results informed the definition of criteria for the evaluation and a case study to complement the theoretical assessment.

The analysis of the results shows that, in general, the expressiveness of the representations ranges from low to moderate -- they scored around 30--40\% in the theoretical evaluation.
Some achieved noteworthy scores in individual areas, but none were adequately functional across all of them, which limits their applicability in general development.
The conclusions apply even to OpenUIDL -- the most flexible representation found; although its capabilities could suffice for prototyping, they will probably not meet the needs of developers.

The findings indicate the need for further progress in user interface representations;
even though the gap between the current and desired capabilities of UI representations is narrow and possible to bridge, it remains unfilled.
New or existing ones can use the study as a point of reference for development.
Other directions of future research include revisions of the list of criteria presented and replicating the study with a broader set of representations.
