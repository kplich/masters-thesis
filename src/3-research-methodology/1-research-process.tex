\section{Research process}\label{sec:research-process}

Figure~\ref{fig:research-process} presents the process of research conducted in the thesis.
Firstly, the papers selected in the literature review were compared and analyzed with regard to techniques enabling the implementation of functionalities related to UIs (task 1).
The results of this analysis -- a list of concepts relevant to UI programming used by selected notations -- are introduced in subsection~\ref{sec:basis-for-evaluation} (task 2.1).

As a result of the analysis, some descriptions (MARIA~\cite{Paterno2009, MariaPDF}, Bouchelligua et al.~\cite{Bouchelligua2010}, Kryštof~\cite{kryvstof2010lpgm}, Achilleos et al.~\cite{Achilleos2011}, and Miao et al.~\cite{Miao2017}) were excluded from the investigation (task 2.2) -- they were not described in enough detail to warrant an accurate evaluation of presented notations.
The exclusion did not negatively influence the research -- these articles did not provide any concepts that were not mentioned in the included ones.
The rest of the notations (Gaouar et al.~\cite{Gaouar2018}, Khan et al.~\cite{Khan2021}, OpenUIDL~\cite{Moldovan2020}, Quid~\cite{molina2018quid, Molina2019}, Soude and Koussonda~\cite{Soude2022}, XANUI~\cite{hermida2016xanui}, Verhaeghe et al.~\cite{Verhaeghe2021visual, Verhaeghe2021behavior}) were included in the research.

Based on the analysis, specific evaluation criteria for evaluating the selected notations (task 3.1.1) were defined; they are presented in section~\ref{sec:evaluation-criteria}.
Specification of a practical experiment in which an example UI is reproduced (task 3.2.1) is described in section~\ref{sec:case-study}.

Finally, the representations were evaluated against the defined criteria (task 3.1.2).
The defined UI was implemented (task 3.2.2) using two publicly available notations (OpenUIDL\furl{https://teleporthq.io/}~\cite{Moldovan2020}, Quid\furl{https://quid.metadev.pro/}~\cite{molina2018quid, Molina2019}).
The model presented by Verhaeghe et al.\furl{https://github.com/badetitou/Casino}~\cite{Verhaeghe2021visual, Verhaeghe2021behavior} was not used because it was only available as a part of a larger project, which posed a technical and logistical challenge.
The completeness of the implementations was then evaluated (task 3.2.3).

\begin{figure}
    \centering
    \includegraphics[width=\textwidth]{./3-research-methodology/research-process}
    \caption{BPMN diagram illustrating the research process.}
    \label{fig:research-process}
\end{figure}

The results of the evaluation and the discussion are included in the next chapter (task 4).
