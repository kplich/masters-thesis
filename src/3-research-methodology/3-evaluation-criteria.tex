\subsection{Evaluation criteria}\label{subsec:evaluation-criteria}

This section elaborates the classification from the previous section and defines the criteria for evaluation of UI descriptions.

\subsubsection{Formulation of criteria}
The list has been compiled based on the concepts found in the papers and completed by referencing various final UI technologies.
The tables below describe aspects of UI programming that should be supported by the representations;
each criterion is additionally numbered for ease of referencing.
Table~\ref{tab:evaluation-criteria-structure} describes evaluation criteria for foundational concepts in GUI development.
Table~\ref{tab:evaluation-criteria-behavior} presents important areas of modelling GUI behavior, complemented by the list of components that representations might support out-of-the-box, presented in table~\ref{tab:evaluation-criteria-components}.
Finally, features used for describing the appearance of GUI are listed in table~\ref{tab:evaluation-criteria-appearance}.
If the feature does not require a further description, it is marked with a dash (\textemdash).

To give a more thorough picture of the capabilities of the notations, some criteria are split into nested criteria;
this is indicated by the numbering level of the criteria.
Top-level criteria are marked with a boldface letter and a number (\textbf{B1}) and separated using a solid line.
Subcriteria are indicated by adding an extra number after the top-level criterion symbol and are written using a regular font (B1.3);
they are separated visually in the table using a dashed line.
The symbol for third-level criteria has another number added and is written using an italicized font (\textit{B1.3.5}).

\subsubsection{Scoring of representations}
To produce the results, each lowest-level criterion (that does not contain subcriteria) in each area was assigned a score of 1 (if the representation fulfilled the criterion) or 0 (otherwise).
The score for higher-level criteria was calculated as an average score for all subcriteria (a sum of scores for each subcriterion divided by the number of subcriteria);
the formula for the score $C_\text{X}$ of a given criterion X is presented as  equation~\ref{eq:3-3-criterion-scoring}.

\begin{equation}
    C_\text{X} =
\begin{dcases}
    \frac{\sum_{i=1}^{n} C_{\text{X.$i$}}}{n} & \text{if X consists of $n$ lower-level criteria X.1, X.2, ..., X.$n$}\\
    1                                         & \text{if the criterion is fulfilled}\\
    0                                         & \text{if the criterion is not fulfilled}
\end{dcases}
    \label{eq:3-3-criterion-scoring}
\end{equation}

The same mechanism was used to calculate a score $S_{\text{Y}}$ for a whole section Y (presented in equation~\ref{eq:3-3-section-scoring}) and a total score ($T_{\text{criteria}}$) for the whole representation (equation~\ref{eq:3-3-total-scoring}).

\begin{equation}
    S_{\text{Y}} = \frac{\sum_{i=1}^{n} C_{\text{Y$i$}}}{n}\text{,}\ \text{where $C_{\text{Y$i$}}$ is the score for criterion Y$i$ and}\ \text{Y} \in \left\{\text{A}, \text{B}, \text{C}, \text{D} \right\}
    \label{eq:3-3-section-scoring}
\end{equation}

\begin{equation}
    T = \frac{S_{\text{A}} + S_{\text{B}} + S_{\text{C}} + S_{\text{D}}}{4}
    \label{eq:3-3-total-scoring}
\end{equation}

This mechanism of scoring allowed for a detailed and unambiguous evaluation while providing a balance in the scoring by treating the criteria equally, regardless of their number of subcriteria.
At the same time, scores for higher-level criteria can be interpreted as a uniform measure of support.

\paragraph{Example of scoring}
Table ~\ref{tab:my-table} presents example results of evaluation of two hypothetical representations (R1 and R2) against a set of criteria.
Fulfilled criteria are marked with a checkmark (\cmark), the rest are marked with a crossmark (\xmark);
scores for higher-level criteria are written as percentages.
The criterion \textbf{A1} is does not contain any subcriteria, therefore it is evaluated directly.
Criterion \textbf{A2} contains two subcriteria, so the score for this criterion is a percentage of subcriteria that fulfill the given criteria \textendash\ in R1, only one of two criteria is satisfied, so the score is 50\%; in R2 both criteria are satisfied, so the score is 100\%.
The same mechanism applies to criteria A3.1, A3.2 and, in fact, the criterion \textbf{A3} \textendash\ its value is calculated as an average of scores for criteria A3.1 and A3.2.
Finally, the same principle of averaging the criteria scores is used to calculate the score for the whole section.

\begin{table}[]
    \small
    \caption{Example results of evaluation (transposed for readability)}
    \label{tab:my-table}
    \begin{tabular}{||c|c|c|ccc|cccccccc||}
        No. & Total  & \textbf{A1} & \textbf{A2} & A2.1       & A2.2   & \textbf{A3} & A3.1   & \textit{A3.1.1} & \textit{A3.1.2} & \textit{A3.1.3} & A3.2    & \textit{A3.2.1} & \textit{A3.2.2} \\
        R1  & 77.8\% & \cmark\ (100\%)  & 50.0\%      & \cmark & \xmark & 83.4\%      & 66.7\% & \xmark          & \cmark          & \cmark          & 100.0\% & \cmark          & \cmark          \\
        R2  & 47.2\% & \xmark\ (0\%)    & 100.0\%     & \cmark & \cmark & 41.7\%      & 33.3\% & \cmark          & \xmark          & \xmark          & 50.0\%  & \cmark          & \xmark
    \end{tabular}
\end{table}

\subsubsection{Architecture and structure}

Table~\ref{tab:evaluation-criteria-structure} describes the criteria for evaluating the structure of the UI;
they mostly can be mapped directly to the concepts from the previous section: application (\textbf{A1}), view (\textbf{A2}), dialogs (\textbf{A3}), assets (\textbf{A4}), and modularization (\textbf{A5}).
The ability to describe the content of the views and components is assumed to be present in all representations, therefore it is not evaluated;
the same applies to the concept of a component.
Only support for using external components (\textbf{A7}) is evaluated in this section.
The classification of components is used to group criteria describing support for predefined components in section \textbf{C}.
Support for custom components is described in criteria describing support for representing the UI behavior (\textbf{B12}).

\begin{longtblr}[
    caption = {Criteria for evaluating the descriptions' ability to describe the structure of the GUI.},
    label = {tab:evaluation-criteria-structure},
]{
    colspec = {cX[2,c]X[3,c]},
    rowhead = 1,
    rows = {m},
}
    \hline[1pt]
    \textbf{No.} & \textbf{Name}                     & \textbf{Description}                               \\*
    \hline[1pt]
    \textbf{A1}  & \textbf{Application}              & a top-level node of the element tree               \\*
    \hline
    \textbf{A2}  & \textbf{View/Screen}              & grouping of related controls                       \\*
    \hline
    \textbf{A3}  & \textbf{Dialogs}                  & support for dialog views                           \\*
    \hline
    \textbf{A4}  & \textbf{Assets}                   & using resources in separate files                  \\*
    \hline
    \textbf{A5}  & \textbf{Modularization}           & splitting the UI description across multiple files \\*
    \hline
    \textbf{A6}  & \textbf{External components}      & using components from third-party sources          \\*
    \hline[1pt]
\end{longtblr}

\subsubsection{Behavior and logic}

Criteria related to behavior of UI are listed below in table~\ref{tab:evaluation-criteria-behavior}.
First two criteria (\textbf{B1} and \textbf{B2}) describe event handling and predefined events emitted by elements.
The table specifies the events in detail and groups them in several categories.
Pointer events are emitted whenever a pointing device (most commonly a mouse, but could be e.g.\ a finger touching a screen) interacts with an element; usually they're emitted by most elements.
So is the case, too, with events related to the state of an element \textendash\ most of them can receive or lose focus, but not only using a mouse, but also e.g.\ using a keyboard or other assistive technologies.
On the other hand, keyboard, form, and input events are only emitted by certain components for which such events are relevant (usually input components).

One of the representations~\cite{Gaouar2018} also included some \enquote{system events} related to the functioning of a mobile device, e.g.\ indicating a incoming or outgoing call or a change in the state of a network connection.
They were omitted from the evaluation, as they fall outside of the scope of UI programming.

Criteria \textbf{B3}\textendash\textbf{B5} describe support for handlers in relatively broad terms.
Further, criteria \textbf{B6} and \textbf{B7} indicate whether the representations support any extensions to handling logic by calling any other services or executing arbitrary code.
These two criteria are rather general, as they fall in the scope of general client-side programming, not in the scope of UI programming.
Criteria \textbf{B8}\textendash\textbf{B11} describe techniques for specifying handlers and describing content in more detail.
Criterion \textbf{B12} lists criteria related to defining custom components.

\begin{longtblr}[
    caption = {Criteria for evaluating the representations' ability to model the behavior of GUIs},
    label = {tab:evaluation-criteria-behavior},
]{
    colspec = {cX[2,c]X[3,c]},
    rowhead = 1,
    rows = {m},
}
    \hline[1pt]
    \textbf{No.}      & \textbf{Name}                               & \textbf{Description}                                                                \\*
    \hline[1pt]
    \textbf{B1}       & \textbf{Handling events}                    & the ability to attach an event handler to a certain UI element                      \\*
    \hline
    \textbf{B2}       & \textbf{Predefined events}                  & \textemdash                                                                         \\*
    \hline[dashed]
    B2.1              & Pointer events                              & events related to a pointing device used - mouse, touch screen, etc.                \\*
    \textit{B2.1.1}   & \textit{mouse down}                         & when a pointer is pressed                                                           \\*
    \textit{B2.1.2}   & \textit{mouse up}                           & when a pointer is released                                                          \\*
    \textit{B2.1.3}   & \textit{click}                              & when a pointer is pressed and released in the same element                          \\*
    \textit{B2.1.4}   & \textit{double click}                       & when a pointer is clicked twice rapidly                                             \\*
    \textit{B2.1.5}   & \textit{long click}                         & when a pointer is pressed for a longed time                                         \\*
    \textit{B2.1.6}   & \textit{mouse enter}                        & when a pointer enters the area of an element                                        \\*
    \textit{B2.1.7}   & \textit{mouse leave}                        & when a pointer leaves the area of an element                                        \\*
    \textit{B2.1.8}   & \textit{mouse move}                         & when a pointer moves within the area of an element                                  \\*
    \textit{B2.1.9}   & \textit{drag}                               & when an element starts being dragged                                                \\*
    \textit{B2.1.10}  & \textit{drop}                               & when an element is being dropped                                                    \\*
    \hline[dashed]
    B2.2              & Keyboard events                             & events caused by clicks from a keyboard                                             \\*
    \textit{B2.2.1}   & \textit{key up}                             & when a key is released                                                              \\*
    \textit{B2.2.2}   & \textit{key down}                           & when a key is pressed                                                               \\*
    \hline[dashed]
    B2.3              & Element state events                        & events related to the state of an element                                           \\*
    \textit{B2.3.1}   & \textit{focus}                              & when an element receives focus                                                      \\*
    \textit{B2.3.2}   & \textit{blur}                               & when an element loses focus                                                         \\*
    \hline[dashed]
    B2.4              & Form and input events                       & events related to form and input elements                                           \\*
    \textit{B2.4.1}   & \textit{submit}                             & when a form is being submitted                                                      \\*
    \textit{B2.4.2}   & \textit{change}                             & when an input element changes its value                                             \\*
    \hline
    \textbf{B3}       & \textbf{Defining handlers}                  & the ability to define an event handler                                              \\
    \hline
    \textbf{B4}       & \textbf{Action types}                       & types of action available                                                           \\*
    B4.1              & Navigation                                  & changing the currently displayed view                                               \\*
    B4.2              & Emitting an event                           & \textemdash                                                                         \\*
    B4.3              & Opening a dialog                            & \textemdash                                                                         \\*
    B4.4              & Closing a dialog                            & \textemdash                                                                         \\*
    B4.5              & Showing a toast                             & \textemdash                                                                         \\*
    B4.6              & Reading a value                             & accessing an element and reading its attributes or state                            \\*
    B4.7              & Writing a value                             & accessing an element and changing its attributes or state                           \\*
    B4.8              & Component method call                       & accessing an element and executing its method                                       \\*
    B4.9              & Throwing an error                           & \textemdash                                                                         \\*
    B4.10             & Composing multiple actions                  & \textemdash                                                                         \\*
    \hline
    \textbf{B5}       & \textbf{Data validation}                    & \textemdash                                                                         \\*
    B5.1              & custom mechanism                            & validation using a dedicated mechanism instead of reusing existing mechanisms       \\*
    B5.2              & during input                                & after every change of the input value                                               \\*
    B5.3              & while submitting                            & before a form is submitted                                                          \\*
    B5.4              & for single input                            & validating constraints for separate form fields                                     \\*
    B5.5              & for groups of inputs                        & validating constraints that encompass multiple form fields                          \\*
    B5.6              & customizing error messages                  & specifying error messages that can be displayed in violation of certain constraints \\*
    \hline
    \textbf{B6}       & \textbf{Integration with external services} & ability to call code beyond the GUI layer                                           \\*
    \hline
    \textbf{B7}       & \textbf{Embedding scripts}                  & ability to use a programming language directly in the UIDL                          \\
    \hline
    \textbf{B8}       & \textbf{Control structures}                 & \textemdash                                                                         \\*
    \hline[dashed]
    B8.1              & within handlers                             & \textemdash                                                                         \\*
    \textit{B8.1.1}   & \textit{conditional execution}              & executing one of multiple actions depending on different boolean conditions         \\*
    \textit{B8.1.2}   & \textit{iteration}                          & executing actions multiple times based on a collection or a loop                    \\*
    \hline[dashed]
    B8.2              & within content                              & \textemdash                                                                         \\*
    \textit{B8.2.1}   & \textit{conditional display}                & displaying content depending on the value of a boolean expression                   \\*
    \textit{B8.2.2}   & \textit{iteration}                          & displaying content multiple times based on a collection or a loop                   \\*
    \hline
    \textbf{B9}       & \textbf{Interpolation}                      & usage of expressions within the content                                             \\*
    \hline
    \textbf{B10}      & \textbf{Data types}                         & types of data available throughout the description                                  \\*
    B10.1             & Boolean                                     & \textemdash                                                                         \\*
    B10.2             & Integer                                     & \textemdash                                                                         \\*
    B10.3             & Float                                       & \textemdash                                                                         \\*
    B10.4             & Decimal                                     & precise representation of a decimal number                                          \\*
    B10.5             & String                                      & \textemdash                                                                         \\*
    B10.6             & Char                                        & \textemdash                                                                         \\*
    B10.7             & Array                                       & an ordered sequence of items                                                        \\*
    B10.8             & Set                                         & an unordered sequence of items                                                      \\*
    B10.9             & Object                                      & an associative array of string keys and values of any type                          \\
    \hline
    \textbf{B11}      & \textbf{Expressions/operators}              & \textemdash                                                                         \\*
    \hline[dashed]
    B11.1             & Arithmetic operators                        & \textemdash                                                                         \\*
    \textit{B11.1.1}  & \textit{addition}                           & \textemdash                                                                         \\*
    \textit{B11.1.2}  & \textit{subtraction}                        & \textemdash                                                                         \\*
    \textit{B11.1.3}  & \textit{multiplication}                     & \textemdash                                                                         \\*
    \textit{B11.1.4}  & \textit{fractional division}                & \textemdash                                                                         \\*
    \textit{B11.1.5}  & \textit{modulo}                             & \textemdash                                                                         \\*
    \textit{B11.1.6}  & \textit{exponentiation}                     & \textemdash                                                                         \\*
    \textit{B11.1.7}  & \textit{equality}                           & \textemdash                                                                         \\*
    \textit{B11.1.8}  & \textit{non-equality}                       & \textemdash                                                                         \\*
    \textit{B11.1.9}  & \textit{greater than}                       & \textemdash                                                                         \\*
    \textit{B11.1.10} & \textit{greater than or equal}              & \textemdash                                                                         \\*
    \textit{B11.1.11} & \textit{less than}                          & \textemdash                                                                         \\*
    \textit{B11.1.12} & \textit{less than or equal}                 & \textemdash                                                                         \\*
    \hline[dashed]
    B11.2             & Boolean operators                           & \textemdash                                                                         \\*
    \textit{B11.2.1}  & \textit{disjunction}                        & \textemdash                                                                         \\*
    \textit{B11.2.2}  & \textit{conjunction}                        & \textemdash                                                                         \\*
    \textit{B11.2.3}  & \textit{negation}                           & \textemdash                                                                         \\*
    \hline[dashed]
    B11.3             & Access operators                            & \textemdash                                                                         \\*
    \textit{B11.3.1}  & \textit{indexing}                           & \textemdash                                                                         \\*
    \textit{B11.3.2}  & \textit{object access}                      & \textemdash                                                                         \\*
    \hline[dashed]
    B11.4             & Ternary operator                            & operator that returns one of two values depending on a boolean condition            \\
    \hline
    \textbf{B12}      & \textbf{Defining custom components}         & features related to defining and describing custom components                       \\*
    B12.1             & content                                     & describing the content of a component                                               \\*
    B12.2             & properties                                  & specifying attributes/properties of a component                                     \\*
    B12.3             & content inputs                              & specifying content as input of component                                            \\*
    B12.4             & custom events                               & emitting custom events from a component                                             \\*
    B12.5             & methods                                     & specifying methods in the component                                                 \\*
    B12.6             & internal state                              & describing the state of the component                                               \\*
    \hline[1pt]
\end{longtblr}

\subsubsection{Predefined components}

Table~\ref{tab:evaluation-criteria-components} lists components that might be predefined in the representation.
Criteria \textbf{C1}\textendash\textbf{C3} describe specialized containers and output components;
plain containers have been included in the subsequent table (\textbf{D1}).
This decision is fairly arbitrary and is caused by the encapsulation of layout in a particular component.
Such an approach blends the two concepts together, making them more difficult to separate from one another.
Specialized components, however, are not merely used to lay out content;
they also provide some structure and functionality independently from their content that is useful in a particular context.
Criteria \textbf{C4} and \textbf{C5} describe input components, in particular \emph{the} input field component and its customizations and criterion \textbf{C6} describes interaction components.

\begin{longtblr}[
    caption = {Criteria for evaluating components predefined by the representations},
    label = {tab:evaluation-criteria-components}
]{
    colspec = {cX[2,c]X[3,c]},
    rowhead = 1,
    rows = {m},
}
    \hline[1pt]
    \textbf{No.}     & \textbf{Name}                          & \textbf{Description}                                                                    \\*
    \hline[1pt]
    \textbf{C1}      & \textbf{Specialized containers}        & containers that provide both layout and functionality                                   \\*
    C1.1             & Tab container                          & allows to switch between multiple views using a tab interface                           \\*
    C1.2             & Web container                          & displays a website                                                                      \\*
    C1.3             & Card                                   & container for a visually and structurally grouping related content                      \\*
    C1.4             & Expander                               & container that can be expanded/contracted to show/hide extra content                    \\*
    C1.5             & Toast                                  & a non-intrusive container providing feedback or communicating important information     \\*
    C1.6             & Menu                                   & a container for additional options for user interaction                                 \\
    \hline
    \textbf{C2}      & \textbf{Simple output components}      & components that display concrete data                                                   \\*
    C2.1             & Image                                  & an element for displaying graphical content                                             \\*
    C2.2             & Video                                  & an element for displaying video content                                                 \\*
    C2.3             & Audio                                  & an element for displaying audio content                                                 \\*
    C2.4             & Text                                   & an element for displaying textual content                                               \\*
    C2.5             & Divider                                & a visual separator for distinguishing sections of content                               \\*
    C2.6             & Simple list                            & list of textual content, ordered or unordered                                           \\*
    C2.7             & Table                                  & arrangement of data in rows and columns                                                 \\*
    C2.8             & Progress display                       & visual representation of the completion of a task                                       \\
    \hline
    \textbf{C3}      & \textbf{Complex output components}     & specialized components that display complex data and might have some internal logic     \\*
    C3.1             & Tree                                   & element for representing hierarchical data                                              \\*
    C3.2             & Carousel                               & component for displaying a set of items in a rotating manner                            \\
    \hline
    \textbf{C4}      & \textbf{Input components}              & components that are used to provide data                                                \\*
    C4.1             & Form                                   & a component for gathering information from users; a container for input fields          \\*
    C4.2             & Input group                            & a grouping of related input fields                                                      \\*
    \hline[dashed]
    C4.3             & Input field types                      & an input field for providing various data                                               \\*
    \textit{C4.3.1}  & \textit{number}                        & \textemdash                                                                             \\*
    \textit{C4.3.2}  & \textit{color}                         & \textemdash                                                                             \\*
    \textit{C4.3.3}  & \textit{date}                          & \textemdash                                                                             \\*
    \textit{C4.3.4}  & \textit{datetime}                      & \textemdash                                                                             \\*
    \textit{C4.3.5}  & \textit{time}                          & \textemdash                                                                             \\*
    \textit{C4.3.6}  & \textit{email}                         & \textemdash                                                                             \\*
    \textit{C4.3.7}  & \textit{file}                          & \textemdash                                                                             \\*
    \textit{C4.3.8}  & \textit{password}                      & \textemdash                                                                             \\*
    \textit{C4.3.8}  & \textit{range}                         & \textemdash                                                                             \\*
    \textit{C4.3.9}  & \textit{search}                        & an input type for providing a search term                                               \\*
    \textit{C4.3.10} & \textit{telephone}                     & \textemdash                                                                             \\*
    \textit{C4.3.11} & \textit{short text}                    & an input type for providing one-line text                                               \\*
    \textit{C4.3.12} & \textit{long text}                     & an input type for providing multi-line text                                             \\*
    \hline[dashed]
    C4.4             & Radio input field                      & single option input                                                                     \\*
    C4.5             & Checkbox                               & element for making a binary selection or choose multiple options from a list            \\*
    C4.6             & Toggle                                 & element for switching between two mutually exclusive states                             \\*
    C4.7             & Single-selection dropdown              & element for choosing a single option from a drop-down list                              \\*
    C4.8             & Multiple-selection dropdown            & element for choosing multiple options from a drop-down list                             \\
    \hline
    \textbf{C5}      & \textbf{Input component customization} & \textemdash                                                                             \\*
    \hline[dashed]
    C5.1             & Constraint types                       & \textemdash                                                                             \\*
    \textit{C5.1.1}  & \textit{pattern}                       & limiting the text input through the means of regular expressions                        \\*
    \textit{C5.1.2}  & \textit{required}                      & requiring that the input is not empty                                                   \\*
    \textit{C5.1.3}  & \textit{max length}                    & limiting the maximum length of provided text                                            \\*
    \textit{C5.1.4}  & \textit{min length}                    & requiring a minimal length of provided text                                             \\*
    \textit{C5.1.5}  & \textit{max}                           & limiting the maximum numeric value of an input                                          \\*
    \textit{C5.1.6}  & \textit{min}                           & requiring a minimum numeric value of an input                                           \\*
    \hline[dashed]
    C5.2             & Component inputs                       & \textemdash                                                                             \\*
    \textit{C5.2.1}  & \textit{disabled}                      & preventing a user from interacting from an input                                        \\*
    \textit{C5.2.2}  & \textit{placeholder}                   & an example text within the input field, providing hints to the user                     \\
    \hline
    \textbf{C6}      & \textbf{Interaction components}        & \textemdash                                                                             \\*
    C6.1             & Button                                 & an element performing a certain function when clicked                                   \\*
    C6.2             & Link                                   & a textual interaction element, usually used for navigation                              \\*
    C6.3             & Floating action button                 & a prominent button used for triggering the most important action within a given context \\*
    \hline[1pt]
\end{longtblr}

\subsubsection{Appearance}
Table~\ref{tab:evaluation-criteria-appearance} lists aspects of appearance that the representations should be able to include in their model.
Criteria \textbf{D1} and \textbf{D2} evaluate the capabilities of describing layouts, including responsive layouts by using media queries.
Criteria \textbf{D3}\textendash\textbf{D4} list expected methods of sizing and positioning of the elements.
Criterion \textbf{D5} lists length units that might be supported by the representations.
Criteria \textbf{D6} and \textbf{D7} describe the most commonly found properties used for styling elements.
Finally, criterion \textbf{D8} checks whether the representations support using CSS directly.

\begin{longtblr}[
    caption = {Criteria for evaluating the representations' ability to describe the appearance of GUIs},
    label = {tab:evaluation-criteria-appearance},
]{
    colspec = {cX[2,c]X[3,c]},
    rowhead = 1,
    rows = {m},
}
    \hline[1pt]
    \textbf{No.} & \textbf{Name}                         & \textbf{Description}                                                                                    \\*
    \hline[1pt]
    \textbf{D1}  & \textbf{Layout containers}            & components that place their content using a certain layout                                              \\*
    D1.1         & Linear layout container               & a layout container that arranges content horizontally or vertically                                     \\*
    D1.2         & Flexible layout (flex) container      & a linear layout container with additional options for children sizing and alignment                     \\*
    D1.3         & Border layout container               & a layout container that positions children in five regions (north, south, east, west, center)           \\*
    D1.4         & Relative layout container             & a layout container, in which children are positioned relative to other children or the container itself \\*
    D1.5         & Grid container                        & a layout container, where children are flexibly positioned in rows and columns                          \\*
    \hline
    \textbf{D2}  & \textbf{Media queries}                & changing the appearance of content based on device characteristics                                      \\*
    \hline
    \textbf{D3}  & \textbf{Sizing methods}               & supported methods of sizing elements                                                                    \\*
    D3.1         & fixed                                 & sizing using absolute units, without relation to the size of the parent component                       \\*
    D3.2         & relative                              & sizing in relation to the parent component or the viewport                                              \\*
    D3.3         & proportional                          & sizing proportional to other children of the component                                                  \\*
    \hline
    \textbf{D4}  & \textbf{Positioning}                  & supported modes of positioning                                                                          \\*
    D4.1         & static                                & default positioning according to the parent component's layout                                          \\*
    D4.2         & absolute                              & positioning at an exact position in the viewport, without regarding the parent layout                   \\*
    D4.3         & relative                              & positioning at an offset position, based on an original position in a container                         \\*
    D4.4         & sticky                                & positioning that fixes an element after content has been scrolled                                       \\
    \hline
    \textbf{D5}  & \textbf{Length units}                 & supported units of length                                                                               \\*
    D5.1         & pixels                                & smallest unit of display on a screen                                                                    \\*
    D5.2         & density-independent pixels            & a physical unit that allows for consistent scaling regardless of the screen size                        \\*
    D5.3         & points                                & smallest unit in typography, 1/72th of an inch                                                          \\*
    D5.4         & percentages                           & percentage of a given dimension of a parent component                                                   \\*
    D5.5         & \texttt{em}                           & font size of the parent element                                                                         \\*
    D5.6         & \texttt{rem}                          & font size of the root element                                                                           \\*
    D5.7         & \texttt{vw}                           & 1\% of the viewport's width                                                                             \\*
    D5.8         & \texttt{vh}                           & 1\% of the viewport's height                                                                            \\*
    \hline
    \textbf{D6}  & \textbf{Basic box model properties}   & properties describing the most common aspects of element appearance                                     \\*
    D6.1         & padding                               & space between the element's border and content                                                          \\*
    D6.2         & margin                                & space around the element                                                                                \\*
    D6.3         & border width                          & \textemdash                                                                                             \\*
    D6.4         & border color                          & \textemdash                                                                                             \\*
    D6.5         & border type                           & \textemdash                                                                                             \\*
    D6.7         & border radius                         & \textemdash                                                                                             \\*
    D6.8         & background color                      & \textemdash                                                                                             \\*
    D6.9         & background image                      & \textemdash                                                                                             \\*
    D6.10        & background gradient                   & \textemdash                                                                                             \\*
    D6.11        & shadow offset                         & \textemdash                                                                                             \\*
    D6.12        & shadow color                          & \textemdash                                                                                             \\*
    D6.13        & shadow blur                           & \textemdash                                                                                             \\
    \hline
    \textbf{D7}  & \textbf{Text styling}                 & properties describing the appearance of text                                                            \\*
    D7.1         & font family                           & \textemdash                                                                                             \\*
    D7.2         & font weight                           & \textemdash                                                                                             \\*
    D7.3         & underlining                           & \textemdash                                                                                             \\*
    D7.4         & italics                               & \textemdash                                                                                             \\*
    D7.5         & font size                             & \textemdash                                                                                             \\*
    D7.6         & text color                            & \textemdash                                                                                             \\*
    \hline
    \textbf{D8}  & \textbf{Embedding stylesheets}        & possibility of using CSS directly or indirectly through files                                           \\*
    \hline[1pt]
\end{longtblr}
